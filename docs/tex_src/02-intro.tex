\chapter*{Введение}
\addcontentsline{toc}{chapter}{Введение}

Хранение изображений --- задача, стоящая перед разработчиками большинства современных информационных систем. Если число фотографий ограничено, то задача не вызывает больших сложностей: хранение фотографий в виде отдельных файлов будет оптимальным решением. Однако существуют информационные системы, для которых заранее неизвестно количество фотографий, потому что это количество может обновляться во время эксплуатации такой системы (увеличиваться). Примером таких информационных систем могут послужить социальные сети и облачные хранилища. В общем случае задача заключается в реализации возможностей хранения и поиска неструктурированных данных: помимо фотографий аналогичная задача решается для хранения аудио и видео файлов, документов, скомпилированных программ и прочих не имеющих четкой структуры данных.

Способ решения задачи зависит от требований к информационной системе. Так, если у системы есть повышенные требования по безопасности, то допускается добавление шифрования данных в обмен на скорость доступа к этим данным. В случае систем, имеющих повышенное ограничение по памяти, может использоваться сжатие данных, которое так же снижает скорость операции поиска. В данной работе будут рассматриваться тип систем, основное требование которых --- наиболее быстрый поиск.

Цель работы --- провести исследование и предложить метод оптимизации операции поиска в существующей NoSQL базе данных.

Чтобы достигнуть поставленной цели, требуется решить следующие задачи:
\begin{itemize}
    \item рассмотреть существующий метод поиска в определенной базе данных;
    \item рассмотреть способы оптимизации поиска;
    \item реализовать программно предложенные способы оптимизации;
    \item провести сравнение результатов.
\end{itemize}
