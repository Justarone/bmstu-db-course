\chapter{Технологическая часть}

В данном разделе представлены средства разработки программного обеспечения, детали реализации и тестирование функций.

\section{Средства реализации}

В качестве языка программирования для реализации кода оптимизации использовался Rust\cite{rust}. Данный выбор обусловлен тем, что проект, поиск в котором требуется оптимизировать, написан на данном языке.

Для тестирования программного обеспечения были использованы инструменты пакетного менеджера Cargo\cite{cargo}, поставляемого вместе с компилятором языка при стандартном способе установке, описанном на официальном сайте языка\cite{rust}. 

В процессе разработки был использован инструмент RLS\cite{rls} (англ. \textit{Rust Language Server}), позволяющий форматировать исходные коды, а также в процессе их написания обнаружить наличие синтаксических ошибок и некоторых логических, таких как, например, нарушение правила владения\cite{rust-learn}.

В качестве среды разработки был выбран текстовый редактор VIM\cite{vim}, поддерживающий возможность установки плагинов\cite{vim-plugins}, в том числе для работы с RLS\cite{rls}.
 %А вообще там на сайте вима есть реклама, и они говорят, что деньги с неё отправляются на помощь детям Уганды, переходите по ссылке!

\section{Реализация алгоритмов}

В листинге \ref{lst:serialize} представлена реализация алгоритма сериализации данных в формат хранения в виде B$^+$-дерева. В листинге \ref{lst:search} представлена реализация алгоритма поиска данных в B$^+$-дереве.

\begin{lstinputlisting}[
        caption={Реализация алгоритма сериализации данных},
        label={lst:serialize},
        style={rust}
    ]{../../src/serialize.rs}
\end{lstinputlisting}

\begin{lstinputlisting}[
        caption={Реализация алгоритма поиска данных},
        label={lst:search},
        style={rust},
    ]{../../src/search.rs}
\end{lstinputlisting}

\section{Функциональные тесты}

В листинге \ref{lst:func} представлены функции для тестирования разработанного кода.

\begin{lstinputlisting}[
        caption={Функциональные тесты},
        label={lst:func},
        style={rust},
    ]{../../src/func.rs}
\end{lstinputlisting}

\section*{Вывод}

В данном разделе были рассмотрены средства, с помощью которых было реализовано ПО, а также представлены листинги кода с реализацией сериализации B$^+$-дерева и поиска в нём. Также были приведены коды, использовавшиеся для тестирования правильности работы кода.
