\chapter{Конструкторская часть}

В данном разделе представлены требования к коду оптимизации, а также схемы алгоритмов, выбранных для решения поставленной задачи.

\section{Требования к коду оптимизации}

Код оптимизации должен предоставлять тот же функционал, что и код, который он заменяет. Учитывая архитектуру базы данных Pearl, требуется реализовать алгоритмы сериализации и поиска данных.

При этом предъявляются следующие требования:
\begin{itemize}
    \item код должен корректно работать и выдавать тот же ответ, что и заменямый код, на любых входных данных;
    \item должны существовать условия, выполнимые на современных компьютерах, при которых код оптимизации будет работать быстрее.
\end{itemize}

\clearpage

\section{Разработка алгоритмов}

\subsection{Алгоритм поиска}

На рисунке \ref{img:search} представлена схема алгоритма поиска данных при хранении в формате B$^+$-дерева.

\imgw{150mm}{search}{Схема алгоритма поиска данных при хранении в формате B$^+$-дерева}

\clearpage

\subsection{Алгоритм сериализации}

На рисунке \ref{img:serialize} представлена схема алгоритма сериализации данных для хранения в формате B$^+$-дерева.

\imgw{150mm}{serialize}{Схема алгоритма сериализации данных для хранения в формате B$^+$-дерева}

\section*{Вывод}

В данном разделе были представлены требования к коду оптимизации и разработаны схемы реализуемых алгоритмов.
